
\documentclass[journal,12pt,twocolumn]{IEEEtran}
%
\usepackage{setspace}
\usepackage{gensymb}
\singlespacing
\usepackage[cmex10]{amsmath}
\usepackage{siunitx}
\usepackage{amsthm}

\usepackage{mathrsfs}

\usepackage{txfonts}
\usepackage{stfloats}

\usepackage{steinmetz}
\usepackage{cite}
\usepackage{cases}
\usepackage{subfig}
\usepackage{longtable}
\usepackage{multirow}
\usepackage{enumitem}
\usepackage{mathtools}
\usepackage{tikz}
\usepackage{circuitikz}
\usepackage{verbatim}
\usepackage{tfrupee}
\usepackage[breaklinks=true]{hyperref}
\usepackage{tkz-euclide} % loads  TikZ and tkz-base
\usetikzlibrary{calc,math}
\usetikzlibrary{fadings}
\usepackage{listings}
    \usepackage{color}                                            %%
    \usepackage{array}                                            %%
    \usepackage{longtable}                                        %%
    \usepackage{calc}                                             %%
    \usepackage{multirow}                                         %%
    \usepackage{hhline}                                           %%
    \usepackage{ifthen}                                           %%
  %optionally (for landscape tables embedded in another document): %%
    \usepackage{lscape}     
\usepackage{multicol}
\usepackage{chngcntr}
\DeclareMathOperator*{\Res}{Res}

\renewcommand\thesection{\arabic{section}}
\renewcommand\thesubsection{\thesection.\arabic{subsection}}
\renewcommand\thesubsubsection{\thesubsection.\arabic{subsubsection}}

\renewcommand\thesectiondis{\arabic{section}}
\renewcommand\thesubsectiondis{\thesectiondis.\arabic{subsection}}
\renewcommand\thesubsubsectiondis{\thesubsectiondis.\arabic{subsubsection}}

\hyphenation{op-tical net-works semi-conduc-tor}
\def\inputGnumericTable{}                                 %%

\lstset{
%language=C,
frame=single, 
breaklines=true,
columns=fullflexible
}
\begin{document}
%


\newtheorem{theorem}{Theorem}[section]
\newtheorem{problem}{Problem}
\newtheorem{proposition}{Proposition}[section]
\newtheorem{lemma}{Lemma}[section]
\newtheorem{corollary}[theorem]{Corollary}
\newtheorem{example}{Example}[section]
\newtheorem{definition}[problem]{Definition}
\newcommand{\BEQA}{\begin{eqnarray}}
\newcommand{\EEQA}{\end{eqnarray}}
\newcommand{\define}{\stackrel{\triangle}{=}}
\bibliographystyle{IEEEtran}
\providecommand{\mbf}{\mathbf}
\providecommand{\pr}[1]{\ensuremath{\Pr\left(#1\right)}}
\providecommand{\qfunc}[1]{\ensuremath{Q\left(#1\right)}}
\providecommand{\sbrak}[1]{\ensuremath{{}\left[#1\right]}}
\providecommand{\lsbrak}[1]{\ensuremath{{}\left[#1\right.}}
\providecommand{\rsbrak}[1]{\ensuremath{{}\left.#1\right]}}
\providecommand{\brak}[1]{\ensuremath{\left(#1\right)}}
\providecommand{\lbrak}[1]{\ensuremath{\left(#1\right.}}
\providecommand{\rbrak}[1]{\ensuremath{\left.#1\right)}}
\providecommand{\cbrak}[1]{\ensuremath{\left\{#1\right\}}}
\providecommand{\lcbrak}[1]{\ensuremath{\left\{#1\right.}}
\providecommand{\rcbrak}[1]{\ensuremath{\left.#1\right\}}}
\theoremstyle{remark}
\newtheorem{rem}{Remark}
\newcommand{\sgn}{\mathop{\mathrm{sgn}}}
\providecommand{\abs}[1]{\left\vert#1\right\vert}
\providecommand{\abs}[1]{\lvert#1\rvert} 
\providecommand{\res}[1]{\Res\displaylimits_{#1}} 
\providecommand{\norm}[1]{\left\lVert#1\right\rVert}
%\providecommand{\norm}[1]{\lVert#1\rVert}
\providecommand{\mtx}[1]{\mathbf{#1}}
\providecommand{\mean}[1]{E\left[ #1 \right]}
\providecommand{\fourier}{\overset{\mathcal{F}}{ \rightleftharpoons}}
%\providecommand{\hilbert}{\overset{\mathcal{H}}{ \rightleftharpoons}}
\providecommand{\system}{\overset{\mathcal{H}}{ \longleftrightarrow}}
	%\newcommand{\solution}[2]{\textbf{Solution:}{#1}}
\newcommand{\solution}{\noindent \textbf{Solution: }}
\newcommand{\cosec}{\,\text{cosec}\,}
\providecommand{\dec}[2]{\ensuremath{\overset{#1}{\underset{#2}{\gtrless}}}}
\newcommand{\myvec}[1]{\ensuremath{\begin{pmatrix}#1\end{pmatrix}}}
\newcommand{\mydet}[1]{\ensuremath{\begin{vmatrix}#1\end{vmatrix}}}
\numberwithin{equation}{subsection}
\makeatletter
\@addtoreset{figure}{problem}
\makeatother
\let\StandardTheFigure\thefigure
\let\vec\mathbf
\renewcommand{\thefigure}{\theproblem}
\def\putbox#1#2#3{\makebox[0in][l]{\makebox[#1][l]{}\raisebox{\baselineskip}[0in][0in]{\raisebox{#2}[0in][0in]{#3}}}}
     \def\rightbox#1{\makebox[0in][r]{#1}}
     \def\centbox#1{\makebox[0in]{#1}}
     \def\topbox#1{\raisebox{-\baselineskip}[0in][0in]{#1}}
     \def\midbox#1{\raisebox{-0.5\baselineskip}[0in][0in]{#1}}
\vspace{3cm}
\title{ASSIGNMENT-5}
\author{R.YAMINI}
\maketitle
\newpage
\bigskip
\renewcommand{\thefigure}{\theenumi}
\renewcommand{\thetable}{\theenumi}
%
\section{QUESTION NO-2.11}
\item If $(x-a)^2+(y-b)^2 = c^2$, for some $c>0$ then prove that $\frac{(1+(y_{1})^2)^\frac{3}{2}}{y_{2}}$ is a constant independent of $a$ and $b$.
%

%
\section{Solution}
Given that $(x-a)^2+(y-b)^2 = c^2$, for some $c>0$.In order to prove $\frac{(1+(y_{1})^2)^\frac{3}{2}}{y_{2}}$ is a constant independent of $a$ and $b$.
Consider,
\begin{align}
    (x-a)^2+(y-b)^2 = c^2
\end{align}
Now differentiating equation (2.0.1) on both sides with respect to $x$  we get
\begin{align}
    2(x-a)+2(y-b)\frac{dy}{dx} = 0
\\ 
y_{1} = \frac{dy}{dx} = \frac{a-x}{y-b}
\end{align}
Differentiating (2.0.3) again with respect to $x$ we have,
\begin{align}
    \frac{d^2y}{dx^2}= \frac{-(y-b)-(a-x)y_{1}}{(y-b)^2}
    \\
    y_{2} = \frac{-(y-b)-(a-x)\brak{\frac{a-x}{y-b}}}{(y-b)^2}
    \\
    = \frac{-(x-a)^2-(y-b)^2}{(y-b)^3}
    \\
    = -\brak{\frac{(x-a)^2+(y-b)^2}{(y-b)^3}}
    \\
   y_{2}= \frac{-c^2}{(y-b)^3}
\end{align}

Now substituting the values of $y_{1}$ from equation (2.0.3) and $y_{2}$ from equation (2.0.8) in 
$\frac{(1+(y_{1})^2)^\frac{3}{2}}{y_{2}}$
\begin{align}
\frac{(1+(y_{1})^2)^\frac{3}{2}}{y_{2}}
&= \frac{(1+\frac{(x-a)^2}{(y-b)^2})^\frac{3}{2}}{\frac{-c^2}{(y-b)^3}}
\\
&= \frac{((x-a)^2+(y-b)^2)^\frac{3}{2}}{(y-b)^3}\brak{\frac{-(y-b)^3}{c^2}}
\end{align}
From equation (2.0.1), we get
\begin{align}
&= \frac{-(c^2)^\frac{3}{2}}{c^2} 
\\
&= -c  
\end{align}
Hence, we have 
$ \frac{(1+(y_{1})^2)^\frac{3}{2}}{y_{2}}= -c$.
\\
Therefore, we have proved that, if $(x-a)^2+(y-b)^2 = c^2$, for some $c>0$ then $\frac{(1+(y_{1})^2)^\frac{3}{2}}{y_{2}}$ is a constant independent of $a$ and $b$.
\end{document}
